%%*****************************************************************************
%% $Id: extex-developers.tex,v 1.1 2005/08/22 07:30:20 gene Exp $
%%*****************************************************************************
%% Author: Gerd Neugebauer
%%-----------------------------------------------------------------------------
\documentclass{extex-doc}

\usepackage{subfigure}

\def\setVersion$#1: #2.#3 ${\gdef\Version{#2.#3}}
\setVersion$Revision: 1.1 $

\newcommand\menu{\textsf}
\newcommand\sub{\(\rightarrow\) }

\begin{document}%%%%%%%%%%%%%%%%%%%%%%%%%%%%%%%%%%%%%%%%%%%%%%%%%%%%%%%%%%%%%%%

\begin{titlepage}
  \parindent=0pt
  \begin{center}
  \vspace*{1pt}
  \vfill
  \ExTeXbox
  \vfill
  \textsf{\bfseries\Huge Developer's Guide}
  \vfill
  \textsf{\Large Version \Version}
  \vfill
  \textsf{\large Gerd Neugebauer}
  \vfill
  \vfill
%\maketitle

  \begin{abstract}\parindent=0pt
    This document describes some basic steps to develop and test
    \ExTeX.  It is meant for newcomers to the project or people who
    want to evaluate \ExTeX\ by inspecting the sources.
  \end{abstract}
  \end{center}
\newpage
\footnotesize
\copyright\ 2005 The \ExTeX\ Group and individual authors listed below 
\medskip

Permission is granted to copy, distribute and/or modify this document
under the terms of the GNU Free Documentation License, Version 1.2 or
any later version published by the Free Software Foundation. A copy of
the license is included in the section entitled ``GNU Free
Documentation License''.
\bigskip

This product includes software developed by the Apache Software
Foundation (http://www.apache.org/).

\vfill

Gerd Neugebauer\\
Im Lerchelsb\"ohl 5\\
64521 Gro\ss-Gerau (Germany)
\smallskip

\href{mailto://gene@gerd-neugebauer.de}{gene@gerd-neugebauer.de}

\end{titlepage}

\tableofcontents
\newpage

%------------------------------------------------------------------------------
\chapter{Introduction}
%@author Gerd Neugebauer

\ExTeX{} aims at providing a high-quality typesetting system. The
development of \ExTeX\ has been inspired by the experiences with \TeX.
The focus lies on an open design and a high degree of configurability.
Thus \ExTeX\ should be a good base for further development.

On the other hand we have to take care not to leave the current user
base of \TeX\ behind. pdf\TeX\ has taught us that a migration path
from \TeX\index{TeX@\TeX} has a positive value in it. In the mean time
the majority of \TeX\ users applies in fact
pdf\TeX\index{pdfTeX@pdf\TeX}.

To provide a backward compatibility of \ExTeX\ with
\TeX\index{TeX@\TeX} one special configuration is provided. Thus
backward compatibility is just a matter of configuration.


\section{Audience}

This document is meant for developers and those interested in the
sources and development processes of \ExTeX. It should contain all
information for getting started quickly.


\section{Mailing Lists}

If you are ready to try \ExTeX{} you might as well want to join a
mailing list to get in contact with the community. The following
mailing lists might be of interest:

\begin{description}
\item[extex@dante.de] \ \\
  A general mailing list about \ExTeX. It has low traffic and is
  mainly in German. Subscribe and unsubscribe via the Web form
  \url{http://www.dante.de/listman/extex}.
\item[extex-en@dante.de] \ \\
  A general mailing list about \ExTeX. It has currently very low
  traffic and is in English. Subscribe and unsubscribe via the Web
  form \url{http://www.dante.de/listman/extex-en}.
\item[extex-devel@dante.de] \ \\
  A mailing list for the exchange of the developers of \ExTeX. It has
  low traffic and is partly in German. Subscribe and unsubscribe via
  the Web form \url{http://www.dante.de/listman/extex-devel}.
\end{description}


\section{Language}

The official project language for \ExTeX\ is English in the US
dialect. The sources are documented in this language and the major
documents are written in this language.

Since some of the developers are German this language might slip in
during intensive discussions.

\section{The Repository}

The sources of \ExTeX\ are stored in a RCS repository. To access this
repository you need access to the internet and RCS installed in some
way.

The coordinates of the repository are:\medskip

\begin{tabular}{ll}\toprule
  Connection type: & \texttt{pserver}			\\
  User:		   & \texttt{anonymous}			\\
  Host:		   & \texttt{cvs.extex.berlios.de}	\\
  Location:	   & \texttt{/cvsroot/extex}		\\
  Module:	   & \texttt{ExTeX}			\\\bottomrule
\end{tabular}\medskip

These coordinates allow you anonymous access to the sources with
reading permissions only.

\section{User Account at Berlios}

To commit changes to the repository you have to be enlisted as a
developer for \ExTeX. A first requirement for this is an account at
Berlios -- the hosting site. 



%------------------------------------------------------------------------------
\chapter{Prerequisites}
%@author Gerd Neugebauer

\section{Java}

You need to have Java 1.4.2 or later installed on your system. You can
get Java for a several systems directly from \url{http://java.sun.com}.
Download and install it according to the installation instructions for
your environment.

Free Java re-implementations are currently not supported. They might
work, but the last time we checked it, GCJ didn't suffice. We would be
happy have someone working on a compatibility layer for a free Java
implementation.


\section{CVS Client}

You need a CVS client installed. In the simplest case this is the
client build into te IDE Eclipse, or the command line version of cvs.


\section{A Command Line Interpreter}

For several tasks it is convenient to have a command line interpreter
at hand. On Unix this can be the (bourne/bash/...) shell. On Windows
we recommend the Cygwin suite which contains the bash.




%------------------------------------------------------------------------------
\chapter{The Development Environment}
%@author Gerd Neugebauer

There is no mandatory IDE for the development of \ExTeX. Nevertheless
in practice you can get good support if you stick to the development
environment widely used within the \ExTeX\ community. This is based on
the Eclipse IDE.

\section{Eclipse}

Eclipse is a free IDE for Java and other programming languages. It
also provides a framework for the development of own programs. But
this is not needed for the \ExTeX\ core. Currently the version 3.1 of
Eclipse is used within the \ExTeX\ development team.
\begin{figure}[h]
  \centering  \includegraphics[scale=.5]{image/eclipse-splash}
  \caption{Eclipse}\label{fig:eclipse}
\end{figure}

\subsection{Eclipse Installation}

Eclipse can be downloaded for free from \url{http://www.eclipse.org}.
There you can get a file appropriate for your operating system
containing the software development kit (SDK). For instance
\begin{description}
\item [eclipse-SDK-3.1-win32.zip]\ \\
  for any decent Windows platform.
\item [eclipse-SDK-3.1-linux-gtk.zip]\ \\
  for Linux on Intel x86 with GTK.
\end{description}

Download the appropriate file and unpack it in the installation
directory. A new subdirectory \texttt{eclipse} will be created
containing all files of Eclipse. You are done with the basic
installation. You can start the \texttt{eclipse} executable found in
the just installed directory.

\begin{figure}[h]
  \centering  \includegraphics[scale=.5]{image/eclipse-workspace}
  \caption{Eclipse Workspace}\label{fig:eclipse-workspace}
\end{figure}
When Eclipse starts you first see the splash screen shown in
figure~\ref{fig:eclipse}. Then Eclipse requests a workspace -- as
shown in figure~\ref{fig:eclipse-workspace}. The workspace is a
directory where the projects live and where your preferences are
stored. If you have chosen the workspace directory carefully, you can
turn on the check mark in this dialog to be not asked again.

Finally you end up in the welcome window of Eclipse shown in
figure~\ref{fig:eclipse-welcome}. Take some time and read the
introductory material found there.
\begin{figure}[h]
  \centering  \includegraphics[scale=.25]{image/eclipse-welcome}
  \caption{Eclipse Initial Window}\label{fig:eclipse-welcome}
\end{figure}

The following sections describe some of the configurations which
should be performed in order to work with Eclipse on \ExTeX.


\subsection{Downloading the Sources}

Now we are ready to create a project for the sources of \ExTeX.
Everything needed can be found in the CVs repository of \ExTeX\ hosted
by Berlios.. Thus we start to get things onto the local host. For this
purpose we need to open a new perspective in Eclipse. A perspective is
a collection of windows which are usually meant for a common task.

A new perspective can be opened via the window \menu{Window \sub Open
  Perspective \sub Other\ldots} which can be seen in
figure~\ref{fig:eclipse-open-perspective-menu}.
\begin{figure}[ht]
  \hbox{}\hfill
  \subfigure[Open Perspective\label{fig:A1}]{%
    \includegraphics[scale=.4]{image/eclipse-open-perspective-menu}}%
  \hfill
  \subfigure[Selecting ``CVS Exploring Perspective''\label{fig:eclipse-open-perspective-menu}]{%
    \includegraphics[scale=.4]{image/eclipse-select-cvs}}%
  \hfill\hbox{}

%  \caption{Open Perspective}\label{fig:eclipse-open-perspective-menu}
\end{figure}

This menu item opens a dialog box which offers some perspectives for
opening. Currently we need a ``CVS Exploring'' perspective. This
perspective is meant for inspecting CVS repositories and manipulation.
Thus this perspective is selected (see
figure~\ref{fig:eclipse-select-cvs}) and the dialog is completed with
the OK button.
%\begin{figure}[ht]
%  \centering  \includegraphics[scale=.4]{image/eclipse-select-cvs}
%  \caption{Selecting ``CVS Exploring Perspective''}\label{fig:eclipse-select-cvs}
%\end{figure}

Now a CVS exploring perspective is opened (see
figure~\ref{fig:eclipse-cvs-1}). You see a lot of windows and icons
there. The tab ``CVS Repositories'' on the left side shows all
repository locations currently known. This list is empty since we have
not added any CVS locations yet.
\begin{figure}[ht]
  \centering  \includegraphics[scale=.25]{image/eclipse-cvs-1}
  \caption{CVS Exploring Perspective}\label{fig:eclipse-cvs-1}
\end{figure}

To add a new repository location press the left mouse button on this
tab and select \menu{New \sub Repository Location\ldots} (see
figure~\ref{fig:eclipse-new-repository}).
\begin{figure}[ht]
  \centering  \includegraphics[scale=.4]{image/eclipse-new-repository}
  \caption{New Repository Location}\label{fig:eclipse-new-repository}
\end{figure}

This brings up the dialog shown in figure~\ref{fig:eclipse-add-cvs}.
Here you can enter the coordinates of the \ExTeX\ CVS repository.
\begin{figure}[ht]
  \centering  \includegraphics[scale=.4]{image/eclipse-add-cvs}
  \caption{The Coordinates of the \ExTeX\ Repository}\label{fig:eclipse-add-cvs}
\end{figure}

Note that you have to enter your account at Berlios ant its password
into the appropriate fields. If you do not have an account you can use
the account name \texttt{anonymous} without any password to get
reading access to the sources.

For this step you need online access to the internet. When the form is
submitted with the OK button, the accessibility of the repository
location is checked. Upon success the new repository location is
added to the list of repository locations as can be seen in figure~\ref{fig:eclipse-cvs}.
\begin{figure}[ht]
  \centering  \includegraphics[scale=.4]{image/eclipse-cvs}
  \caption{The \ExTeX\ Repository Listed}\label{fig:eclipse-cvs}
\end{figure}


\subsection{Configuring Eclipse}

Eclipse can be configured in a wide range. In the following sections
some configuration options are proposed for the seamless development
of \ExTeX. The configuration is performed via the preferences dialog.
This dialog can be opened via \menu{Window \sub Preferences\ldots}
\begin{figure}[ht]
  \centering  \includegraphics[scale=.4]{image/eclipse-preferences}
  \caption{Eclipse Preferences}\label{fig:eclipse-preferences}
\end{figure}


\subsection{The Code Formatter}

Eclipse comes has a code formatter which can be invoked easily. This
code formatter can be configured for different needs. A configuration
for \ExTeX\ is contained in the repository under
\texttt{develop/eclipse/formatter.xml}.



\subsection{Checkstyle}

Checkstyle is a tool for checking the adherence of Java source code to
certain rules. The rules can be freely configured. The \ExTeX\
repository contains a set of rules for checkstyle.

Checkstyle comes in a command line version and as a plug-in for
Eclipse. This plugin has to be installed first.


\subsection{Spelling}

Since English is not the native language of each developer it is a
good idea to enable the spell checking of the source code. This
feature is provided by Eclipse. In figure~ref{fig:eclipse-spelling}
you can seen the preference page where you can activate the spell
checking and provide a dictionary.
\begin{figure}[th]
  \centering
  \includegraphics[scale=.4]{image/spelling}
  \caption{Spelling Preferences}\label{fig:eclipse-spelling}
\end{figure}

A dictionary can be got from SCOWL
(\url{http://wordlist.sourceforge.net/}). You might want to use the US
dictionary of medium size. Since this contains enough words to fit but
not too much obscure words which hide typos.

After the spell checking is activated potential typos are marked in
the editor with yellow lines. Correction proposals can be requested
with the quick fix shortcut Ctrl-1.


\subsection{Compiling \ExTeX}

Any source file in Eclipse is compiled automatically when the file is
saved. Thus it is usually not necessary to compile things manually.
If you feel the need to recompile everything you can achieve this by
selecting \menu{Project \sub Clean\ldots} while the item \menu{Project
\sub Build Automatically} is checked (see figure~\ref{fig:eclipse-recompile}).
\begin{figure}[th]
  \centering
  \includegraphics[scale=.4]{image/eclipse-recompile}
  \caption{Recompiling a Project}\label{fig:eclipse-recompile}
\end{figure}

Another recompilation can be triggered via the ant task \texttt{compile}.

\subsection{Running \ExTeX}

\ExTeX\ can be run from within Eclipse.

\INCOMPLETE

\subsection{Committing Changes}

\INCOMPLETE


%------------------------------------------------------------------------------
\section{Command Line Use}
%@author Gerd Neugebauer


\subsection{Downloading the Sources}

You need to download the sources of \ExTeX. On the command line this
can be done with the following commands:

\begin{verbatim}
% cvs -d:pserver:anonymous@cvs.extex.berlios.de:/cvsroot/extex login
% cvs -z3 -d:pserver:anonymous@cvs.extex.berlios.de:/cvsroot/extex co ExTeX
\end{verbatim}


\subsection{Checkstyle}

Checkstyle is a source code checker.
\ExTeX\ should show a homogeneous appearance of the sources.

\subsection{Ant}

Apache Ant is a build system for Java. It can be considered state of
the art for Java programs to come with ant scripts. \ExTeX\ supports
ant by providing a \texttt{build.xml} file for various tasks.

The files needed for running ant are included in the \ExTeX\
repository. Thus no additional installation is required.


\subsection{JUnit}

JUnit is the state of the art concerning test automation for Java
programs. 

\INCOMPLETE


\subsection{Compiling \ExTeX}

\INCOMPLETE


\subsection{Running \ExTeX}

\INCOMPLETE


%------------------------------------------------------------------------------
\section{Use with Emacs and JDE}
%@author Gerd Neugebauer

\INCOMPLETE



%------------------------------------------------------------------------------
\section{Modelling: Jude}

Jude is a UML modeller written in Java. It is distributed in a
community edition for free use. Currently the version 1.6.2 is
available from \url{http://www.esm.jp/jude-web/index.html}. 
\begin{figure}[thp]
  \centering
  \includegraphics[width=\textwidth]{image/jude-seq}
  \caption{Jude}\label{fig:jude}
\end{figure}

Jude should be used for any situations where UML diagrams are needed.
A screenshot of Jude can be seen in figure~\ref{fig:jude}.


%------------------------------------------------------------------------------
\chapter{The Source Tree Organization}
%@author Gerd Neugebauer

\section{The Toplevel Directory}

The toplevel directory of an \ExTeX\ project contains certain files
and sub-directories.

\begin{description}
\item[develop] \ \\
  The sub-directory \texttt{develop} contains bits and pieces needed
  for development only.
\item[doc]  \ \\
  The sub-directory \texttt{doc} contains documentation -- papers
  written by the \ExTeX\ Group and material collected from elsewhere.
\item[lib]  \ \\
  The sub-directory \texttt{lib} contains libraries which need to be
  present for the final executable to run.
\item[src]  \ \\
  The sub-directory \texttt{src} contains the sources.
\item[target]  \ \\
  The sub-directory \texttt{target} contains the generated files. This
  directory is not present in the CVS archive.
\item[tmp]  \ \\
  The sub-directory \texttt{tmp} may contain intermediary files. This
  directory is not present in the CVS archive.
\item[util]  \ \\
  The sub-directory \texttt{util} contains some utilities.
\item[work]  \ \\
  The sub-directory \texttt{work} may contain working files of single
  developers. It is not shared via the repositiory.
\item[www]  \ \\
  The sub-directory \texttt{www} contains the sources for te Web pages
  of \ExTeX.
\end{description}

\section{User's Working Area}

\INCOMPLETE



%------------------------------------------------------------------------------
\chapter{Unit Tests}
%@author Gerd Neugebauer

\INCOMPLETE




%------------------------------------------------------------------------------
\chapter{The Web Pages}
%@author Gerd Neugebauer

\INCOMPLETE



%------------------------------------------------------------------------------
\chapter{Licenses for \ExTeX}
%@author Gerd Neugebauer

\INCOMPLETE


%------------------------------------------------------------------------------
\appendix
%------------------------------------------------------------------------------
\chapter{Licenses}
\input{fdl}


\end{document}%%%%%%%%%%%%%%%%%%%%%%%%%%%%%%%%%%%%%%%%%%%%%%%%%%%%%%%%%%%%%%%%%
%
% Local Variables: 
% mode: latex
% TeX-master: nil
% End: 
