%%*****************************************************************************
%% $Id: extex-developers.tex,v 1.4 2005/09/04 11:34:58 gene Exp $
%%*****************************************************************************
%% Author: Gerd Neugebauer
%%-----------------------------------------------------------------------------
\documentclass{extex-doc}

\usepackage{subfigure}

\def\setVersion$#1: #2.#3 ${\gdef\Version{#2.#3}}
\setVersion$Revision: 1.4 $

\newcommand\menu{\textsf}
\newcommand\sub{\(\rightarrow\) }
\newcommand\File[1]{\texttt{#1}}

\newcommand\Macro[1]{\texttt{\char`\\ #1}}

\makeatletter
\renewcommand\maketitle{
  \begin{center}
  \parindent=0pt
  \vspace*{1pt}
  \vfill
  \ExTeXbox
  \vfill
  \textsf{\bfseries\Huge \@title}
  \vfill
  \textsf{\Large Version \Version}
  \vfill
  \textsf{\large \@author}
  \vfill
  \vfill
  \end{center}
}
\makeatother


\begin{document}%%%%%%%%%%%%%%%%%%%%%%%%%%%%%%%%%%%%%%%%%%%%%%%%%%%%%%%%%%%%%%%

\begin{titlepage}\parindent=0pt

  \title{Developer's Guide}
  \author{Gerd Neugebauer}
  \maketitle

  \begin{center}
    \begin{abstract}\parindent=0pt
      This document describes some basic steps to develop and test
      \ExTeX.  It is meant for newcomers to the project or people who
      want to evaluate \ExTeX\ by inspecting the sources.
    \end{abstract}
  \end{center}
  \newpage
  \footnotesize
  \copyright\ 2005 The \ExTeX\ Group and individual authors listed below 
  \medskip

Permission is granted to copy, distribute and/or modify this document
under the terms of the GNU Free Documentation License, Version 1.2 or
any later version published by the Free Software Foundation. A copy of
the license is included in the section entitled ``GNU Free
Documentation License''.
\bigskip

This product includes software developed by the Apache Software
Foundation (\url{http://www.apache.org/}).

\vfill

Gerd Neugebauer\\
Im Lerchelsb\"ohl 5\\
64521 Gro\ss-Gerau (Germany)
\smallskip

\href{mailto://gene@gerd-neugebauer.de}{gene@gerd-neugebauer.de}

\end{titlepage}

\tableofcontents
\newpage

%------------------------------------------------------------------------------
\chapter{Introduction}
%@author Gerd Neugebauer

\ExTeX{} aims at providing a high-quality typesetting system. The
development of \ExTeX\ has been inspired by the experiences with \TeX.
The focus lies on an open design and a high degree of configurability.
Thus \ExTeX\ should be a good base for further development.

On the other hand we have to take care not to leave the current user
base of \TeX\ behind. pdf\TeX\ has taught us that a migration path
from \TeX\index{TeX@\TeX} has a positive value in it. In the mean time
the majority of \TeX\ users applies in fact
pdf\TeX\index{pdfTeX@pdf\TeX}.

To provide a backward compatibility of \ExTeX\ with
\TeX\index{TeX@\TeX} one special configuration is provided. Thus
backward compatibility is just a matter of configuration.


\section{Audience}

This document is meant for developers and those interested in the
sources and development processes of \ExTeX. It should contain all
information for getting started quickly.


\section{Mailing Lists}

If you are ready to try \ExTeX{} you might as well want to join a
mailing list to get in contact with the community. The following
mailing lists might be of interest:

\begin{description}
\item[extex@dante.de] \ \\
  A general mailing list about \ExTeX. It has low traffic and is
  mainly in German. Subscribe and unsubscribe via the Web form
  \url{http://www.dante.de/listman/extex}.
\item[extex-eng@dante.de] \ \\
  A general mailing list about \ExTeX. It has currently very low
  traffic and is in English. Subscribe and unsubscribe via the Web
  form \url{http://www.dante.de/listman/extex-eng}.
\item[extex-devel@dante.de] \ \\
  A mailing list for the exchange of the developers of \ExTeX. It has
  low traffic and is partly in German. Subscribe and unsubscribe via
  the Web form \url{http://www.dante.de/listman/extex-devel}.
\end{description}


\section{Organizational Agreements}

The developers of \ExTeX\ have agreed on some rule for cooperation.
Those rules are documented here.

\subsection{Language}

The official project language for \ExTeX\ is English in the US
dialect. The sources are documented in this language and the major
documents are written in this language.

Since some of the developers are German this language might slip in
during intensive discussions.


\subsection{Maintainers of Files}

Each file has a single maintainer -- even if there are several
authors. The maintainer has to be informed and has to agree on any
changes in the file. The maintainership is usually indicated in the
Java sources with the help of te tag \texttt{@author}. The first
author is always the maintainer.

Changes to a file can be carried out by the maintainer or delegated to
somebody else. The maintainer can change if both the old and new
maintainer agree in this.




%------------------------------------------------------------------------------
\chapter{Prerequisites}
%@author Gerd Neugebauer

\section{User Account at Berlios}

To commit changes to the repository you have to be enlisted as a
developer for \ExTeX. A first requirement for this is an account at
Berlios -- the hosting site. If you just want to read the sources then
you can use anonymous access.



\section{Java}

You need to have Java 1.4.2\index{Java} or later installed on your
system. You can get Java for a several systems directly from
\url{java.sun.com}. Download and install it according to the
installation instructions for your environment.

To check that you have an appropriate Java on your path you can use
the command \texttt{java} with the argument \texttt{-version}. This
can be seen in the following listing:

\lstset{morecomment=[l]{\#}}%
\begin{lstlisting}{morecomment=[l][keywordstyle]{>}}
# java -version
java version "1.4.2_09"
Java(TM) 2 Runtime Environment, Standard Edition (build 1.4.2_09-b03)
Java HotSpot(TM) Client VM (build 1.4.2_09-b03, mixed mode)
#
\end{lstlisting}

Free Java implementations are currently not supported. They might
work, but the last time we checked it, GCJ didn't suffice. We would be
happy to have someone working on a compatibility layer for a free Java
implementation.


\section{TEXMF}
%@author Gerd Neugebauer

If you want to use more than the pure \ExTeX\ engine, fonts and macros
can be inherited from a texmf tree\index{texmf}. \ExTeX\ itself does
not contain a full texmf tree. It comes just with some rudimentary
files necessary for testing. Thus you should have installed a texmf
tree, e.g. from a \TeX Live\index{TeXlive@\TeX Live} installation.
This can be found on the \href{http://www.ctan.org}{Comprehensive
  \TeX\ Archive Network (CTAN)}\index{CTAN}.

There is no need to install the texmf tree in a special place. You
have to tell \ExTeX\ anyhow where it can be found. It is even possible
to work with several texmf trees.

One requirement for the texmf trees is that they have a file database
(\File{ls-R}). \ExTeX\ can be configured to work without it, but then
\ExTeX\ is deadly slow. Thus you do not really want to try this
alternative.


\section{CVS Client}

You need a CVS client installed. In the simplest case this is the
client build into the IDE Eclipse, or the command line version of cvs.


\section{A Command Line Interpreter}

For several tasks it is convenient to have a command line interpreter
at hand. On Unix this can be the (bourne, bash,\ldots) shell. On Windows
we recommend the Cygwin suite which contains the bash.





%------------------------------------------------------------------------------
\chapter{The Development Environment}
%@author Gerd Neugebauer

There is no mandatory IDE for the development of \ExTeX. Nevertheless
in practice you can get good support if you stick to the development
environment widely used within the \ExTeX\ community. This is based on
the Eclipse IDE.

\section{Eclipse}

Eclipse is a free IDE for Java and other programming languages. It
also provides a framework for the development of own programs. But
this is not needed for the \ExTeX\ core. Currently the version 3.1 of
Eclipse is used within the \ExTeX\ development team.
\begin{figure}[h]
  \centering  \includegraphics[scale=.5]{image/eclipse-splash}
  \caption{Eclipse}\label{fig:eclipse}
\end{figure}

\subsection{Eclipse Installation}

Eclipse can be downloaded for free from \url{http://www.eclipse.org}.
There you can get a file appropriate for your operating system
containing the software development kit (SDK). For instance
\begin{description}
\item [eclipse-SDK-3.1-win32.zip]\ \\
  for any decent Windows platform.
\item [eclipse-SDK-3.1-linux-gtk.zip]\ \\
  for Linux on Intel x86 with GTK.
\end{description}

Download the appropriate file and unpack it in the installation
directory. A new subdirectory \texttt{eclipse} will be created
containing all files of Eclipse. You are done with the basic
installation. You can start the \texttt{eclipse} executable found in
the just installed directory.

\begin{figure}[h]
  \centering  \includegraphics[scale=.5]{image/eclipse-workspace}
  \caption{Eclipse Workspace}\label{fig:eclipse-workspace}
\end{figure}
When Eclipse starts you first see the splash screen shown in
figure~\ref{fig:eclipse}. Then Eclipse requests a workspace -- as
shown in figure~\ref{fig:eclipse-workspace}. The workspace is a
directory where the projects live and where your preferences are
stored. If you have chosen the workspace directory carefully, you can
turn on the check mark in this dialog to be not asked again.

Finally you end up in the welcome window of Eclipse shown in
figure~\ref{fig:eclipse-welcome}. Take some time and read the
introductory material found there.
\begin{figure}[h]
  \centering  \includegraphics[scale=.33]{image/eclipse-welcome}
  \caption{Eclipse Initial Window}\label{fig:eclipse-welcome}
\end{figure}

The following sections describe some of the configurations which
should be performed in order to work with Eclipse on \ExTeX.


\subsection{Downloading the Sources}

Now we are ready to create a project for the sources of \ExTeX.
Everything needed can be found in the CVs repository of \ExTeX\ hosted
by Berlios.. Thus we start to get things onto the local host. For this
purpose we need to open a new perspective in Eclipse. A perspective is
a collection of windows which are usually meant for a common task.

A new perspective can be opened via the window \menu{Window \sub Open
  Perspective \sub Other\ldots} which can be seen in
figure~\ref{fig:eclipse-open-perspective-menu}. This menu item opens a
dialog box which offers some perspectives for opening. Currently we
need a ``CVS Exploring'' perspective. This perspective is meant for
inspecting CVS repositories and manipulation. Thus this perspective is
selected (see figure~\ref{fig:eclipse-select-cvs}) and the dialog is
completed with the OK button.
\begin{figure}[ht]
  \hbox{}\hfill
  \subfigure[Open Perspective\label{fig:eclipse-open-perspective-menu}]{%
    \includegraphics[scale=.4]{image/eclipse-open-perspective-menu}}%
  \hfill
  \subfigure[Selecting ``CVS Exploring Perspective''\label{fig:eclipse-select-cvs}]{%
    \includegraphics[scale=.4]{image/eclipse-select-cvs}}%
  \hfill\hbox{}

  \caption{Switching to a Perspective}\label{fig:eclipse-perspective}
\end{figure}

Now a CVS exploring perspective is opened (see
figure~\ref{fig:eclipse-cvs-1}). You see a lot of windows and icons
there. The tab ``CVS Repositories'' on the left side shows all
repository locations currently known. This list is empty since we have
not added any CVS locations yet.
\begin{figure}[htbp]
  \centering  \includegraphics[scale=.33]{image/eclipse-cvs-1}
  \caption{CVS Exploring Perspective}\label{fig:eclipse-cvs-1}
\end{figure}

To add a new repository location press the left mouse button on this
tab and select \menu{New \sub Repository Location\ldots} (see
figure~\ref{fig:eclipse-new-repository}). This brings up the dialog
shown in figure~\ref{fig:eclipse-add-cvs}. Here you can enter the
coordinates of the \ExTeX\ CVS repository.
\begin{figure}[htp]
  \hbox{}\hfill
  \subfigure[New Repository Location\label{fig:eclipse-new-repository}]{%
    \includegraphics[scale=.4]{image/eclipse-new-repository}}%
  \hfill
  \subfigure[The Coordinates of the \ExTeX\ CVS
  Repository\label{fig:eclipse-add-cvs}]{%
    \includegraphics[scale=.4]{image/eclipse-add-cvs}}%
  \caption{Adding the \ExTeX\ CVS Repository}
\end{figure}

Note that you have to enter your account at Berlios and its password
into the appropriate fields. If you do not have an account you can use
the account name \texttt{anonymous} without any password to get
reading access to the sources.

For this step you need online access to the internet. When the form is
submitted with the OK button, the accessibility of the repository
location is checked. Upon success the new repository location is
added to the list of repository locations as can be seen in
figure~\ref{fig:eclipse-cvs}.
\begin{figure}[htp]
  \hbox{}\hfill
  \subfigure[The \ExTeX\ Repository Listed\label{fig:eclipse-cvs}]{%
    \includegraphics[scale=.4]{image/eclipse-cvs}}%
  \hfill
  \subfigure[Selecting to check-out of
  \ExTeX\label{fig:eclipse-checkout-extex}]{%
    \includegraphics[scale=.4]{image/eclipse-checkout-extex}}%
  \caption{Checking-out of \ExTeX}
\end{figure}

The next step consists of the check-out of the sources into an Eclipse
project. To accomplish this you have to open the repository location
and the HEAD within. Right-click the item \texttt{ExTeX} in the list
(see figure~\ref{fig:eclipse-cvs}) and select \menu{Checkout} in the
appearing context menu (see figure~\ref{fig:eclipse-checkout-extex}).
This will instruct Exclipe to create a new project in the workspace
and fill it with the files from the repository.

Eclipse shows a progress bar during the check-out (see
figure~ref{fig:eclipse-checkout}). This operation may take some time
-- we have been really busy creating files. When the checkout is
finished you will find the project \texttt{ExTeX} in Eclipse
containing the files within. The appearance of the Package View with
those files is shown in figure~\ref{fig:eclipse-extex-project}.
\begin{figure}[htp]
  \hbox{}\hfill
  \subfigure[The Checkout Progress Bar\label{fig:eclipse-checkout}]{%
    \includegraphics[scale=.4]{image/eclipse-checkout}}%
  \hfill
  \subfigure[The \ExTeX\ Project in the Package
  View\label{fig:eclipse-extex-project}]{%
    \includegraphics[scale=.4]{image/eclipse-extex-project}}%
  \caption{Checking out \ExTeX\ from the Repository}
\end{figure}

\subsection{Configuring Eclipse}

Eclipse can be configured in a wide range. In the following sections
some configuration options are proposed for the seamless development
of \ExTeX. The configuration is performed via the preferences dialog.
This dialog can be opened via \menu{Window \sub Preferences\ldots}
\begin{figure}[htp]
  \hbox{}\hfill
  \subfigure[Eclipse Preferences\label{fig:eclipse-preferences}]{%
    \includegraphics[scale=.4]{image/eclipse-preferences}}%
  \hfill
  \subfigure[Eclipse Print Margin\label{fig:eclipse-print-margin}]{%
    \includegraphics[scale=.4]{image/eclipse-print-margin}}%
  \caption{Some simple Settings}
\end{figure}

This menu brings up a dialog with many tabs which can be used to
adjust the behaviour of Eclipse in many ways. The first step described
below consists of the adaption of the appearance of the text editors.
In the tree view on the left side of the dialog select \menu{General
  \sub Editors \sub Text Editors} as shown in
figure~\ref{fig:eclipse-print-margin}. 

Now you can adjust some values on the right side of the dialog. Set
the tag width to 8. Check the item \menu{Show print margin}. Adjust
the print margin to 80. And finally change the print color to red. The
settings are stored in the workspace by accepting the settings with
the \menu{OK} button.

The rational is that the tabs should be used in the traditional sense
of eight chracters wide. In fact this is just a fallback. Usually tabs
should be avoided where possible. The print margin of 80 is a weak
rule. Try to limit yourself to this width. Sometimes it is not
reasonable. Thus the checkstyle rules allow some more characters
before complaining.

The following sections describe some more of the configuration
options. You should really consider to follow the instructions to make
maximal use of the configurations provided with \ExTeX.



\subsection{Code Templates}

Code templates provide a convenient way of filling in a frame for the
documentation whenever some code is generated by Eclipse. The \ExTeX\ 
repository contains in the file
\File{develop/eclipse/codetemplates.xml} some definitions of code
templates. To import those definitions use the preferences (see
figure~\ref{fig:eclipse-preferences}). Here select the item \menu{Java
  \sub Code Style \sub Code Templates}. The button \menu{Import\ldots}
can leads to a file selector where the file
\File{develop/eclipse/codetemplates.xml} should be entered.
\begin{figure}[htp]
  \centering  \includegraphics[scale=.4]{image/eclipse-templates}
  \caption{Eclipse Preferences}\label{fig:eclipse-templates}
\end{figure}

After the code templates have been loaded a minor adaption is
required. The entry under the key \menu{Comments \sub Types} containes
hard-wired a name and email address of the author. Here the own name
and email address should be entered.


\subsection{The Code Formatter}

Eclipse comes has a code formatter which can be invoked easily. This
code formatter can be configured for different needs. A configuration
for \ExTeX\ is contained in the repository under
\texttt{develop/eclipse/formatter.xml}.

\INCOMPLETE

The formatter for Ant files has distinct parameters which should be
adapted. The Prference page can be found under the key \menu{Ant \sub
  Editor \sub Formatter}. The values should be adjusted as shown in
figure~\ref{fig:eclipse-ant-formatter}.
\begin{figure}[htp]
  \centering  \includegraphics[scale=.4]{image/eclipse-ant-formatter}
  \caption{Settings for the  Ant Formatter}\label{fig:eclipse-ant-formatter}
\end{figure}


\subsection{Checkstyle}

Checkstyle is a tool for checking the adherence of Java source code to
certain rules. The rules can be freely configured. The \ExTeX\
repository contains a set of rules for checkstyle.

Checkstyle comes in a command line version and as a plug-in for
Eclipse. This plugin has to be installed first.

The configuration can be found in
\texttt{develop/eclipse/checkstyle-plugin.xml}.

\INCOMPLETE


\subsection{Spelling}

Since English is not the native language of each developer it is a
good idea to enable the spell checking of the source code. This
feature is provided by Eclipse. In figure~ref{fig:eclipse-spelling}
you can seen the preference page where you can activate the spell
checking and provide a dictionary.
\begin{figure}[htp]
  \centering
  \includegraphics[scale=.4]{image/spelling}
  \caption{Spelling Preferences}\label{fig:eclipse-spelling}
\end{figure}

A dictionary can be got from SCOWL
(\url{http://wordlist.sourceforge.net/}). You might want to use the US
dictionary of medium size. Since this contains enough words to fit but
not too much obscure words which hide typos.

After the spell checking is activated potential typos are marked in
the editor with yellow lines. Correction proposals can be requested
with the quick fix shortcut Ctrl-1.


\subsection{Compiling \ExTeX}

Any source file in Eclipse is compiled automatically when the file is
saved. Thus it is usually not necessary to compile things manually. If
you feel the need to recompile everything you can achieve this by
selecting \menu{Project \sub Clean\ldots} while the item \menu{Project
  \sub Build Automatically} is checked (see
figure~\ref{fig:eclipse-recompile}).
\begin{figure}[thp]
  \centering
  \includegraphics[scale=.4]{image/eclipse-recompile}
  \caption{Recompiling a Project}\label{fig:eclipse-recompile}
\end{figure}

Another recompilation can be triggered via the ant task \texttt{compile}.

\subsection{Running \ExTeX}

\ExTeX\ can be run from within Eclipse. We will describe here the
execution of the compiled sources from a workspace. The executio of an
external program would be an alterative. But this is only of minor
relevance for a developer.

\INCOMPLETE

\subsection{Committing Changes}

Eclipse ships with a CVS plugin which hides the details of the
underlying version control system. Thus things are quite simple for
the newcomers. On the other hand they are differnt from the procedure
on the command line or other tools whic mimic the command line (like
WinCVS or TortoiseCVS).

The metaphor used in Eclipse is the synchronisation of the workspace
with the reporsitory. In the course of this syncronisation changes in
the workspace files are committed to the repository, changes from the
repository are updated into the workspace, and conflicts can be
resolved. The conflict resolution -- also known as merging -- is the
demanding task. Thus it has to be performed by a human.

To start the synchronisation select in the \menu{Package Manager} or
\menu{Navigator} view the topmost \ExTeX\ node and activate in the
context menu (right mouse button) the entry \menu{Team \sub Synchronize
with Repository} (see figure~\ref{fig:eclipse-team}).
\begin{figure}[htp]
  \centering
  \includegraphics[scale=.4]{image/eclipse-team}
  \caption{Starting Synchronization}\label{fig:eclipse-team}
\end{figure}



\INCOMPLETE



\subsection{Running Ant from within Eclipse}\label{sec:eclipse.ant}

To use Ant from within Eclipse you have to open the Ant view. This can
be acomplished via the menu \menu{Window \sub Show View \sub Ant} (see
figure~\ref{fig:eclipse-ant-open}).

In this view use the leftmost tool to add an Ant file. In the file
selector choose the file \File{ExTeX/build.xml}. The Ant file is added
to the (previously empty) list. It can be open to show the Ant target
available (see figure~\ref{fig:eclipse-ant}).
\begin{figure}[htp]
  \hbox{}\hfill
  \subfigure[Opening an Ant View\label{fig:eclipse-ant-open}]{%
    \includegraphics[scale=.4]{image/eclipse-ant-open}}%
  \hfill
  \subfigure[The Ant View for \ExTeX\label{fig:eclipse-ant}]{%
    \includegraphics[scale=.4]{image/eclipse-ant}}%
  \caption{Ant in Eclipse}
\end{figure}

A double click on a target starts it's execution. The output is shown
in a Console view.

A description of the targets can be found in section~\ref{sec:Ant}.


\subsection{Creating Javadoc}

To create the Javadoc HTML description of the sources you can use the
Ant target \texttt{javadoc}. See sections \ref{sec:eclipse.ant} and
\ref{sec:Ant}. The result can be found in the directory
\File{target/javadoc}.


\subsection{Creating the Installer}

To create the inastaller you can use the Ant target
\texttt{installer}. See sections \ref{sec:eclipse.ant} and
\ref{sec:Ant}. The result can be found in the file
\File{target/ExTeX-setup.jar}.


%------------------------------------------------------------------------------
\section{Command Line Use}
%@author Gerd Neugebauer




\subsection{Downloading the Sources}


The sources of \ExTeX\ are stored in a RCS repository. To access this
repository you need access to the internet and RCS installed in some
way.

The coordinates of the repository are:\medskip

\begin{tabular}{ll}\toprule
  Connection type: & \texttt{pserver}			\\
  User:		   & \texttt{anonymous}			\\
  Host:		   & \texttt{cvs.extex.berlios.de}	\\
  Location:	   & \texttt{/cvsroot/extex}		\\
  Module:	   & \texttt{ExTeX}			\\\bottomrule
\end{tabular}\medskip

These coordinates allow you anonymous access to the sources with
reading permissions only.

You need to download the sources of \ExTeX. On the command line this
can be done with the following commands:

\begin{lstlisting}{}
% cvs -d:pserver:anonymous@cvs.extex.berlios.de:/cvsroot/extex login
% cvs -z3 -d:pserver:anonymous@cvs.extex.berlios.de:/cvsroot/extex co ExTeX
\end{lstlisting}{}

If you want to participate in the development and are enlisted at
Berlios you should use your account and password instead of the
anonymous account.


\subsection{Checkstyle}

Checkstyle is a source code checker.
\ExTeX\ should show a homogeneous appearance of the sources. Thus
certain rules should be followed. Some of the rules are checked by the
following command:

\begin{lstlisting}{}
% build checkstyle
\end{lstlisting}{}

The result can be found in the file \File{target/checkstyle.txt}.

\subsection{Ant}\label{sec:Ant}

Apache Ant (\url{http://ant.apache.org}) is a build system for Java.
It can be considered state of the art for Java programs to come with
Ant scripts. \ExTeX\ supports ant by providing a \texttt{build.xml}
file for various tasks.

The files needed for running Ant are included in the \ExTeX\
repository. Thus no additional installation is required. Just some
setting need to be performed before Ant can be used.

An environment variable \verb|JAVA_HOME| should be defined which points
to the JDK. The following jars should be placed on the environment
variable \verb|CLASSPATH|:
\begin{itemize}
\item \verb|$JAVA_HOME/lib/tools.jar|
\item \verb|$JAVA_HOME/lib/classes.zip|
\item and all jars found in \File{develop/lib}
\end{itemize}

The Ant can be invoked like in

\begin{lstlisting}{}
  $JAVA_HOME/bin/java -Dant.home=$ANT_HOME org.apache.tools.ant.Main compile
\end{lstlisting}

These steps are performed by the shell script \File{build} in the
\ExTeX\ directory. Thus you can achieve the same effect -- without any
preparations except setting \verb|JAVA_HOME| -- with the following command:

\begin{lstlisting}{}
  build compile
\end{lstlisting}

The Ant configuration can be found in the file \File{build.xml} in the
\ExTeX\ main directory. This configuration contains at least the
following targets:

\begin{description}
\item [all] This target builds nearly everything.
\item [compile] Tis target compiles all Java files of the sources into
  the directora \File{target/classes}. Note, that the test classess
  are not compiled.
\item [jar] This target arranges that the file \File{target/extex.jar}
  is created. It contains the compiled sources of  \ExTeX.
\item [javadoc] This target creates the Javadoc HTML files in the
  directory \File{target/javadoc}.
\item [checkstyle] This target performs the checkstyle tests and
  creates a report in \File{target/checkstyle.txt}.
\item [tests] This target aplies all JUnit test cases.
\item [installer] This target creates the installer. The result is
  placed in the file \File{target/ExTeX-setup.jar}.
\item [clean] This target deletes some generated files.
\end{description}


\subsection{JUnit}

JUnit is the state of the art concerning test automation for Java
programs. Thus \ExTeX\ provides some test cases in the form of JUnit
classes.

\INCOMPLETE

\begin{lstlisting}{}
  build tests
\end{lstlisting}


\subsection{Compiling \ExTeX}

Compiling \ExTeX form the command line can be accomplished with the
help of the build script. The build script is a wrapper around Ant. It
can be invoked with the following command:

\begin{lstlisting}{}
  build compile
\end{lstlisting}

The generated files are placed in the sub-directory
\texttt{target/classes}. Thus if this directory and the jars in
\File{lib} are on the class path then \ExTeX\ can be run immediately.


\subsection{Running \ExTeX}

\ExTeX\ can be run with the help of the \ExTeX\ script in the main
directory or by a direct invocation of Java. The start script is
provided for Unix under the name \File{extex} and for Windows under
the name \File{extex.bat}.
\begin{lstlisting}{}
  extex sample.tex
\end{lstlisting}{}

For the usual purposes these scripts can be used as a plug-in
replacement for \TeX. See the User's Guide for the command line
options.

To run \ExTeX\ from the command line prepare the class path -- i.e.
the environment variable \texttt{CLASSPATH} -- to contain all
libraries found in the directory \texttt{lib}. In addition the
directory \texttt{target/classes} have to be on the class path.
Then you can invoke \ExTeX\ like in the following example:

\begin{lstlisting}{}
  java de.dante.extex.Main.TeX sample.tex
\end{lstlisting}{}

The command line arguments are the same as for \File{extex} mentioned
above.


\subsection{Creating Javadoc}

Creating the Javadoc HTML pages can best be complished with the help
of the build script. Here the target \texttt{javadoc} does everythign
necessary:

\begin{lstlisting}{}
  build javadoc
\end{lstlisting}{}

As the result of this invocation the Javadoc HTML pages are stored in
the sub-directory \File{target/javadoc}.


\subsection{Creating the Installer}

The installer can be build with the help of the build script. The
invocation looks as follows:

\begin{lstlisting}{}
  build installer
\end{lstlisting}{}

After the work is complete the installer can be found in the file
\File{ExTeX-setup.jar} in the directory \File{target}. The use of the
installer is described in the User's Manual.

Alternatively the installer can also be created with the Ant task
\texttt{installer}. Using this method can be applied from the command
line and from within Eclipse.

Note that the installer is automatically created once a day and
provided in the snapshot directory of the \ExTeX\ Web site.


%------------------------------------------------------------------------------
\section{Use with Emacs and JDEE}
%@author Gerd Neugebauer

JDEE is the extension of Emcas for the development with Java. \ExTeX\
contais soem support files for use in this context.

\INCOMPLETE



%------------------------------------------------------------------------------
\section{Modelling: Jude}

Jude is a UML modeller written in Java. It is distributed in a
community edition for free use. Currently the version 1.6.2 is
available from \url{http://www.esm.jp/jude-web/index.html}. 
\begin{figure}[thp]
  \centering
  \includegraphics[width=\textwidth]{image/jude-seq}
  \caption{Jude}\label{fig:jude}
\end{figure}

Jude should be used for any situations where UML diagrams are needed.
A screenshot of Jude can be seen in figure~\ref{fig:jude}.

Models for \ExTeX\ should be placed in the directory \File{doc/models}.


%------------------------------------------------------------------------------
\chapter{Source Code Documentation}
%@author Gerd Neugebauer

The source code has to be documented. \TeX\ shows us a good example of
a proper documenatation. Donald Knuth has invented the Web system to
keep together the documentation and the source code. The source code
and documentation are extracted from a common file. In the Java world
the Javadoc system has been invented for a similar purpose. 

\section{Javadoc}

The Javadoc conventions for comments make it possible to extract the
relevant part of the documentation and generate several outut formats
from it. The primary output format is HTML.
\begin{figure}[tbh]
  \centering
  \includegraphics[scale=.33]{image/javadoc}
  \caption{Javadoc in the Browser}\label{fig:eclipse-javadoc}
\end{figure}

\section{Documentation of Primitives}

\INCOMPLETE




%------------------------------------------------------------------------------
\chapter{The Source Tree Organization}
%@author Gerd Neugebauer


In this section the description of the directory hierarchy is
contained. This structure is oriented on the structuring proposed by
Maven (\url{http://maven.apache.org}).


\section{The Toplevel Directory}

The toplevel directory of an \ExTeX\ project contains certain files
and sub-directories.

\begin{description}
\item[develop] \ \\
  The sub-directory \texttt{develop} contains bits and pieces needed
  for development only.
\item[doc]  \ \\
  The sub-directory \texttt{doc} contains documentation -- papers
  written by the \ExTeX\ Group and material collected from elsewhere.
\item[lib]  \ \\
  The sub-directory \texttt{lib} contains libraries which need to be
  present for the final executable to run.
\item[src]  \ \\
  The sub-directory \texttt{src} contains the sources.
\item[target]  \ \\
  The sub-directory \texttt{target} contains the generated files. This
  directory is not present in the CVS archive.
\item[tmp]  \ \\
  The sub-directory \texttt{tmp} may contain intermediary files. This
  directory is not present in the CVS archive.
\item[util]  \ \\
  The sub-directory \texttt{util} contains some utilities for
  development. They are not included into the installer.
\item[work]  \ \\
  The sub-directory \texttt{work} may contain working files of single
  developers. It is not shared via the repositiory.
\item[www]  \ \\
  The sub-directory \texttt{www} contains the sources for the Web pages
  of \ExTeX.
\end{description}

\section{\texttt{develop}: Development Support}

\begin{description}
\item[eclipse] 
\item[lib] 
\end{description}

\section{\texttt{doc}: Documentation}

\begin{description}
\item[DevelopersGuide] 
\item[Library] 
\item[Publications] 
\item[UsersGuide] 
\item[models] 
\item[notes] 
\end{description}

\section{\texttt{lib}: Third-Party Libraries}

This directory contains libraries needed for \ExTeX\ to run.

\section{\texttt{src}: The Sources}

\begin{description}
\item[font] 
\item[java] 
\item[javadoc] 
\item[text] 
\end{description}

\section{\texttt{util}: Utilities}

This directory contains vairous utitlies and scripts.

\begin{description}
\item[Installer] 
\end{description}

\section{\texttt{work}: User's Working Area}

Any user may have some files in the project area, Those files should
not be committed to the Repository. For this purpose the directory
\texttt{work} is reserved.

\section{\texttt{www}: The Web Site}


\begin{description}
\item[src] 
\end{description}



%------------------------------------------------------------------------------
\chapter{Design Details}
%@author Gerd Neugebauer

This chapter contains some explanations, tips \& tricks. It might be
helpful to read them when you are concerned with the related topics.


\InputIfFileExists{tmp/howto}{}



%------------------------------------------------------------------------------
\chapter{Unit Tests}
%@author Gerd Neugebauer

\INCOMPLETE




%------------------------------------------------------------------------------
\chapter{The Web Pages}
%@author Gerd Neugebauer

\section{Overview}

\ExTeX\ has a domain of its own. This domain \url{www.extex.org} has
been registered by DANTE e.V. In this location the official Web pages
(see figure~\ref{fig:www-extex-org}) are provided.
\begin{figure}[htb]
  \centering
  \includegraphics[scale=.4]{image/www-extex-org}
  \caption{www.extex.org}\label{fig:www-extex-org}
\end{figure}

The Web pages are build with a simple generator for a Web site written
in Perl. It has been made for \ExTeX. The aim is the ease of
maintainance for normal content of pages. They are stored as simple
HTML files and augmented automatically upon publication.

The layout is separated form the content and stored in several files.
This makes it very easy to adapt the appearance without touching the
contents.

The sources are kept in the subdirectory \texttt{src}. The generated
files are put into the subdirectory \texttt{www}. Both locations can
be configured.

To generate the Web site run the following command, where the current
directory is the directory \texttt{www}:

\begin{lstlisting}{}
  make
\end{lstlisting}{}

This command creates a complete directory hierarchy with all necessary
sub-di\-rec\-to\-ries in \texttt{../target/www}. An exception are the
directories named \texttt{CVS}. Those directories are ignored.

The files starting with \verb|.| or ending in \verb|~| or in
\texttt{.bak} are also ignored. The files not ending in \verb|.html|
are copied into the destination tree. The files ending in \verb|.html|
are processed as follows: Text is inserted before the \verb|</head>|
tag from the file \File{.headEnd}. Text is inserted after the
\verb|<body>| tag from the file \File{.bodyStart}. Text is inserted
before the \verb|</body>| tag from the file \File{.bodyEnd}.

The text to be inserted is sought in the current directory and in case
of failure upwards in the super-directories until it is found. In the
inserted files the following entities and tags are replaced:

\begin{description}
\item [\tt\&top;]\ \\
  this is the relative path to the top directory.
\item [\tt\&year;]\ \\
  this is the current year when generating.
\item [\tt\&month;]\ \\
  this is the current month when generating.
\item [\tt\&day;]\ \\
  this is the current day when generating.
\item [\tt<tabs/>]\ \\
  this is replaced by the contents of the file .tabs.
\item [\tt<navigation/>]\ \\
  this is replaced by the contents of the file .navigation.
\item [\tt<info/>]\ \\
  this is replaced by the contents of the file .info.
\end{description}

Note, that even so it looks like XML processing, currently the
processing is based on string manipulation. Thus tricks possible with
XML might not work here.

\section{Layout}

The current layout has the scheme shown in figure~\ref{fig:www-layout}.
\definecolor{bg}{gray}{.95}
\begin{figure}[htbp]
  \centering

  \definecolor{bg}{gray}{.95}
  \definecolor{shadow}{gray}{.8}
  \begin{pgfpicture}{-1.25mm}{-5mm}{132mm}{51mm}
    
    \begin{pgftranslate}{\pgfpoint{22mm}{41mm}}
      \begin{pgftranslate}{\pgfpoint{1mm}{-1mm}}
        \color{shadow}
        \pgfmoveto{\pgfpoint{0mm}{0mm}} \pgflineto{\pgfpoint{5mm}{10mm}}
        \pgflineto{\pgfpoint{105mm}{10mm}} \pgflineto{\pgfpoint{100mm}{0mm}}
        \pgflineto{\pgfpoint{0mm}{0mm}} \pgffillstroke
      \end{pgftranslate}
      \color{bg}
      \pgfmoveto{\pgfpoint{0mm}{0mm}} \pgflineto{\pgfpoint{5mm}{10mm}}
      \pgflineto{\pgfpoint{105mm}{10mm}} \pgflineto{\pgfpoint{100mm}{0mm}}
      \pgflineto{\pgfpoint{0mm}{0mm}} \pgffillstroke
      \color{black}
      \pgfmoveto{\pgfpoint{0mm}{0mm}} \pgflineto{\pgfpoint{5mm}{10mm}}
      \pgflineto{\pgfpoint{105mm}{10mm}} \pgflineto{\pgfpoint{100mm}{0mm}}
      \pgflineto{\pgfpoint{0mm}{0mm}} \pgfstroke
      \pgfputat{\pgfpoint{52.5mm}{5mm}}{\pgfbox[center,center]{\sf\itshape Header}}
    \end{pgftranslate}
    
    \begin{pgftranslate}{\pgfpoint{18.5mm}{34mm}}
      \begin{pgftranslate}{\pgfpoint{1mm}{-1mm}}
        \color{shadow}
        \pgfmoveto{\pgfpoint{0mm}{0mm}} \pgflineto{\pgfpoint{2.5mm}{5mm}}
        \pgflineto{\pgfpoint{102.5mm}{5mm}} \pgflineto{\pgfpoint{100mm}{0mm}}
        \pgflineto{\pgfpoint{0mm}{0mm}} \pgffillstroke
      \end{pgftranslate}
      \color{bg}
      \pgfmoveto{\pgfpoint{0mm}{0mm}} \pgflineto{\pgfpoint{2.5mm}{5mm}}
      \pgflineto{\pgfpoint{102.5mm}{5mm}} \pgflineto{\pgfpoint{100mm}{0mm}}
      \pgflineto{\pgfpoint{0mm}{0mm}} \pgffillstroke
      \color{black}
      \pgfmoveto{\pgfpoint{0mm}{0mm}} \pgflineto{\pgfpoint{2.5mm}{5mm}}
      \pgflineto{\pgfpoint{102.5mm}{5mm}} \pgflineto{\pgfpoint{100mm}{0mm}}
      \pgflineto{\pgfpoint{0mm}{0mm}} \pgfstroke
      \pgfputat{\pgfpoint{52.5mm}{2.5mm}}{\pgfbox[center,center]{\sf\itshape Tab Bar}}
    \end{pgftranslate}
    
    \begin{pgftranslate}{\pgfpoint{-1mm}{2mm}}
      \begin{pgftranslate}{\pgfpoint{1mm}{-1mm}}
        \color{shadow}
        \pgfmoveto{\pgfpoint{0mm}{0mm}} \pgflineto{\pgfpoint{15mm}{30mm}}
        \pgflineto{\pgfpoint{35mm}{30mm}} \pgflineto{\pgfpoint{20mm}{0mm}}
        \pgflineto{\pgfpoint{0mm}{0mm}} \pgffillstroke
      \end{pgftranslate}
      \color{bg}
      \pgfmoveto{\pgfpoint{0mm}{0mm}} \pgflineto{\pgfpoint{15mm}{30mm}}
      \pgflineto{\pgfpoint{35mm}{30mm}} \pgflineto{\pgfpoint{20mm}{0mm}}
      \pgflineto{\pgfpoint{0mm}{0mm}} \pgffillstroke
      \color{black}
      \pgfmoveto{\pgfpoint{0mm}{0mm}} \pgflineto{\pgfpoint{15mm}{30mm}}
      \pgflineto{\pgfpoint{35mm}{30mm}} \pgflineto{\pgfpoint{20mm}{0mm}}
      \pgflineto{\pgfpoint{0mm}{0mm}} \pgfstroke
      \pgfputat{\pgfpoint{17.75mm}{15mm}}{\pgfbox[center,center]{\parbox{20mm}{\centering\sf\itshape
            Navigation\\Area~~~}}}
    \end{pgftranslate}
    
    \begin{pgftranslate}{\pgfpoint{22mm}{2mm}}
      \begin{pgftranslate}{\pgfpoint{1mm}{-1mm}}
        \color{shadow}
        \pgfmoveto{\pgfpoint{0mm}{0mm}} \pgflineto{\pgfpoint{15mm}{30mm}}
        \pgflineto{\pgfpoint{75mm}{30mm}} \pgflineto{\pgfpoint{60mm}{0mm}}
        \pgflineto{\pgfpoint{0mm}{0mm}} \pgffillstroke
      \end{pgftranslate}
      \color{bg}
      \pgfmoveto{\pgfpoint{0mm}{0mm}} \pgflineto{\pgfpoint{15mm}{30mm}}
      \pgflineto{\pgfpoint{75mm}{30mm}} \pgflineto{\pgfpoint{60mm}{0mm}}
      \pgflineto{\pgfpoint{0mm}{0mm}} \pgffillstroke
      \color{black}
      \pgfmoveto{\pgfpoint{0mm}{0mm}} \pgflineto{\pgfpoint{15mm}{30mm}}
      \pgflineto{\pgfpoint{75mm}{30mm}} \pgflineto{\pgfpoint{60mm}{0mm}}
      \pgflineto{\pgfpoint{0mm}{0mm}} \pgfstroke
      \pgfputat{\pgfpoint{37.5mm}{15mm}}{\pgfbox[center,center]{\sf\itshape
          Content Area}}
    \end{pgftranslate}
    
    \begin{pgftranslate}{\pgfpoint{85mm}{2mm}}
      \begin{pgftranslate}{\pgfpoint{1mm}{-1mm}}
        \color{shadow}
        \pgfmoveto{\pgfpoint{0mm}{0mm}} \pgflineto{\pgfpoint{15mm}{30mm}}
        \pgflineto{\pgfpoint{35mm}{30mm}} \pgflineto{\pgfpoint{20mm}{0mm}}
        \pgflineto{\pgfpoint{0mm}{0mm}} \pgffillstroke
      \end{pgftranslate}
      \color{bg}
      \pgfmoveto{\pgfpoint{0mm}{0mm}} \pgflineto{\pgfpoint{15mm}{30mm}}
      \pgflineto{\pgfpoint{35mm}{30mm}} \pgflineto{\pgfpoint{20mm}{0mm}}
      \pgflineto{\pgfpoint{0mm}{0mm}} \pgffillstroke
      \color{black}
      \pgfmoveto{\pgfpoint{0mm}{0mm}} \pgflineto{\pgfpoint{15mm}{30mm}}
      \pgflineto{\pgfpoint{35mm}{30mm}} \pgflineto{\pgfpoint{20mm}{0mm}}
      \pgflineto{\pgfpoint{0mm}{0mm}} \pgfstroke
      \pgfputat{\pgfpoint{17.5mm}{15mm}}{\pgfbox[center,center]{\parbox{20mm}{\centering\sf\itshape
            Info\\Area~~~}}}
    \end{pgftranslate}
    
    \begin{pgftranslate}{\pgfpoint{-1.25mm}{-5mm}}
      \begin{pgftranslate}{\pgfpoint{1mm}{-1mm}}
        \color{shadow}
        \pgfmoveto{\pgfpoint{0mm}{0mm}} \pgflineto{\pgfpoint{2.5mm}{5mm}}
        \pgflineto{\pgfpoint{102.5mm}{5mm}} \pgflineto{\pgfpoint{100mm}{0mm}}
        \pgflineto{\pgfpoint{0mm}{0mm}} \pgffillstroke
      \end{pgftranslate}
      \color{bg}
      \pgfmoveto{\pgfpoint{0mm}{0mm}} \pgflineto{\pgfpoint{2.5mm}{5mm}}
      \pgflineto{\pgfpoint{102.5mm}{5mm}} \pgflineto{\pgfpoint{100mm}{0mm}}
      \pgflineto{\pgfpoint{0mm}{0mm}} \pgffillstroke
      \color{black}
      \pgfmoveto{\pgfpoint{0mm}{0mm}} \pgflineto{\pgfpoint{2.5mm}{5mm}}
      \pgflineto{\pgfpoint{102.5mm}{5mm}} \pgflineto{\pgfpoint{100mm}{0mm}}
      \pgflineto{\pgfpoint{0mm}{0mm}} \pgfstroke
      \pgfputat{\pgfpoint{52.5mm}{2.5mm}}{\pgfbox[center,center]{\sf\itshape Footer}}
    \end{pgftranslate}
  \end{pgfpicture}

  \caption{Layout of the Web pages}\label{fig:www-layout}
\end{figure}

The Header contains the right aligned Logo only.
It is the same on all pages.
The Tab Bar shows the topmost navigation items with the Tab metophor.
The Navigation Area shows all navigation items.
It is the same on all pages.
The Info Area shows some info items specific for the current
navigation item.

The Content Area contains the contents of the page. This is maintained
by the site authors. The Footer contains a simple copyrigt note.


\section{Automatic Generation}

The web pages are generated automatically every night. This task is
performed with the help of a cron job on shell.berlios.de under the
account gene. In the course of this generation the current sources are
checked out from te CVS repository

Thus the normal user simply has to edit the pages in the area
\texttt{www/src} and check them into the CVS repository. The rest
happens automagically.


%------------------------------------------------------------------------------
\chapter{Licenses for \ExTeX}
%@author Gerd Neugebauer

\INCOMPLETE


%------------------------------------------------------------------------------
\appendix
%------------------------------------------------------------------------------
\chapter{Licenses}
\input{fdl}


\end{document}%%%%%%%%%%%%%%%%%%%%%%%%%%%%%%%%%%%%%%%%%%%%%%%%%%%%%%%%%%%%%%%%%
%
% Local Variables: 
% mode: latex
% TeX-master: nil
% End: 
