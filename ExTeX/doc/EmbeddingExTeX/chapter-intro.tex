%%*****************************************************************************
%% $Id: chapter-intro.tex,v 1.1 2006/03/08 10:29:56 gene Exp $
%%*****************************************************************************
%% Author: Gerd Neugebauer
%%-----------------------------------------------------------------------------

\chapter{Introduction}

\ExTeX\ has been designed with  several principals in mind. One of the
design goals is to provide a highly configurable system. This system
is composed from several components whch can be recombined to achieve
a variety of effects.

Thus the components of \ExTeX\ can be assembled to fulfill different
needs. In this tutorial we will explore how to use \ExTeX\ in various
ways.

\section{Components}

Oe of the design principles of \ExTeX\ is the use of components. A
component is a unit of the software which provides a certian
functionality defined by one or more interfaces. On the other side it
may depend on the services of other components.

Interfaces play a central role.

\section{Dependency Injection}

Another design principle of \ExTeX\ is called dependency injection --
somtimes also referred to as inversion of control. This design
principle states that the dependencies of a component are passed to
the component.

In frameworks like J2EE a naming service can be used by a component to
acquire references to other components needed to fulfill the own
service. This is the inverse approach to dependency injection.


\section{Getting \ExTeX}


%\section{}


%
% Local Variables: 
% mode: latex
% TeX-master: "embedExTeX.tex"
% End: 
